\documentclass[12pt,letterpaper,fleqn]{report}

\usepackage[top=1in,bottom=1in,left=1in,right=1in]{geometry}
\usepackage{multicol}
\usepackage{graphicx}
\usepackage{pdfpages}
\usepackage{amsmath}
\usepackage{verbatim}
\usepackage[margin=10pt,font=small,labelfont=bf]{caption}
\usepackage{empheq}
\usepackage{hyperref}

\newcommand*\widefbox[1]{\fbox{\hspace{2em}#1\hspace{2em}}}
\newcommand{\dC}{^{o}C}
\newcommand{\HRule}{\noindent\rule{\linewidth}{0.5mm}}

\setlength{\parindent}{0in} % No indentation
\setlength{\parskip}{0.5cm}
\setlength{\tabcolsep}{4pt} % Tight table

\setcounter{secnumdepth}{2}

\providecommand{\e}[1]{\ensuremath{\times 10^{#1}}}
\renewcommand{\thesection}{Part \Roman{section}.}

%\title{Guam Renewable Integration}
%\subtitle{ENERGY 291 Final Report}
%\author{Martin Chang \and Ryan Satterlee}
%\date{\today}

\begin{document}

%\maketitle

\begin{titlepage}
  \begin{center}

    \HRule \\[0.4cm]

    \textsc{\LARGE Guam Renewable Integration}\\[0.2cm]
    \textsc{\large ENERGY 291 Final Project}
    
    
    \HRule \\[1.5cm]

    \begin{minipage}{0.45\textwidth}
      \begin{flushleft} \large
        \emph{Authors:}\\
        Martin \textsc{Chang}\\
        Ryan \textsc{Satterlee}
      \end{flushleft}
    \end{minipage}
    \begin{minipage}{0.5\textwidth}
      \begin{flushright} \large
        \emph{Professor:}\\
        Adam \textsc{Brandt}\\%[3cm]
        \emph{}
      \end{flushright}
    \end{minipage}

    \vfill
    {\large \today}

  \end{center}
\end{titlepage}

\section{Introduction}

High electricity prices on island nations and territories result from
a limitation of conventional energy resources, necessitating the
expensive imports of primarily liquid fuels by sea. The potential for
cheaper systems exists through a combination of increased deployment
of renewable energy, energy storage, and optimal grid
planning. Additionally, reducing reliance on imported fuels increases
the robustness and resiliency of the grid system against fuel
shortages and price shocks.

In this report, we will investigate the optimization of grid planning
on the island of Guam, with a focus on minimizing net present value
costs of a system upgrade. This will involve simultaneously optimizing
two interdependent variables: (1) the quantity of new renewable energy
capacity to build and (2) the geolocation of these new systems, based
on recommended locations by NREL, resource availability, and
transmission costs.

Guam is currently serviced by a single utility, the Guam Power
Authority (GPA), which makes island wide grid changes and planning
more feasible. GPA owns 552.8 MW of gross generation capacity, 663
miles of T\&D lines, and 29 substations \cite{gpa14a}. On an
annual basis, Guam consumes on the order of 2 billion kWh, generated
by means of petroleum products, primarily residual fuel oil (RFO \#6)
and diesel (No. 2 distillate) [1, 2]. This electricity comes at a cost
of \$0.27/kWh as of January 2012, or about 2.5 times that of the US
mainland. Additionally, Guam is considering exploring LNG as a
conventional alternative, and passed a renewable portfolio standard of
5\% by 2015 and 25\% by 2035 \cite{eia12}.

Although it is well accepted that there is the potential for
economically sound investments in new energy technologies,
particularly solar PV, on islands that import oil for electricity,
this project seeks to develop how this plan might be carried out and
to determine what the limiting constraints on the problem are.

\section{Methods}

The model is a nonlinear cost minimization. The nonlinearity arises
from discontinuties due to transmission costs, which are fully
incurred for a site upon first construction at that site. Other costs
include capital costs (a function of installed capacity), fixed
operation \& maintenance (O\&M) costs (a function of cumulative
installed capacity), and variable O\&M costs (a function of dispatched
generation. 

Objective Function:
  \[\text{min} \quad  \sum_{s \in SITES}\sum_{t \in TIME}(C_{st}^T + C_s^Cx_{st}
  + C_s^FB_{st} + C_s^VD_{st}) / (1 + d)^t\]
  Subject to:

  RPS Goal:
  \[\forall y \in YEARS \sum_{r \in RENEWABLES}\sum_{t \in TIME} D_{rt} \ge RPS_y \times
  \sum_{s \in SITES}\sum_{t \in TIME} D_{st}\]

  Dispatch Limit:
  \[\forall s,t \in SITES, TIME: D_{st} \le B_{st}\]

  Availability:
  \[\forall r,t \in RENEWABLES, TIME: D_{rt} \le A_{rt}B_{rt}\]

  Load:
  \[\forall t \in TIME: \sum_{s \in SITES} D_{st} = L_{t}\]

  Capacity:
  \[\forall r,t \in RENEWABLES, TIME: B_{rt} \le P_r^{max}\]

  Positivity Constraints:
  \[\forall  \ge 0\]
  
  where:

  {\setlength{\parindent}{-1em}
    $\quad x_{st}^{}$ is the Decision variable: instantaneous amount of resource to build [$MW$]\\
    $\quad B_{st}^{}$ is the Defined variable: cumulative developed resource [$MW$]\\
    $\quad C_{s}^{F}$ is the Fixed O\&M Cost + Transmission Costs [$\$/MWh$]\\
    $\quad C_{s}^{C}$ is the Capital Cost [$\$/MW$]\\
    $\quad C_{st}^{T}$ is the Tranmission Capital Cost [$\$/MW$]\\
    $\quad C_{s}^{V}$ is the Variable O\&M Costs [$\$/MW$]\\
    $\quad L_{ht}^{}$ is the Load [$MWh$]\\
    $\quad D_{st}^{}$ is the Dispatch [$MWh$]\\
    $\quad S_{t}^{}$ is the RPS goal [$\%$]\\
    $\quad A_{st}^{}$ is the Resource availability [$MW/km^2$]\\
    $\quad P_{s}^{max}$ is the Maximum site capacity [$MW$]\\
    $\quad d$ is the Discount rate\\
    $\quad SITES$ is the set of all generation sites (Oil, Diesel,
    Solar, and Wind)\\
    $\quad RENEWABLES$ is the set of all renewable generation sites
    (Solar and Wind)\\
    $\quad TIME$ is the set of all timesteps\\
    $\quad YEARS$ is the est of all years (2015 - 2036)
  }



\section{Data}

Our project will be broken into several stages in order to increase
the complexity and power of the model over time as time and data
availability permits. First, we will be optimizing the placement of
the renewable energy generation needed to meet Guam’s RPS (5\% by 2015
up to 25\% by 2035) given current grid constraints. This will
primarily rely on the geospatial locations and capacities of existing
grid infrastructure, resource maps of renewable energy sources, and
costs associated with building and operating these plants and the
transmission network. Additional work could involve optimizing the
grid infrastructure with new construction, which would require
evaluating the impact of land features (such as slope, urban areas,
etc.) on the cost of new T\&D lines. Finally, we would require the
demand profiles for electricity in Guam.

An important source of information for these data sets may be Guam
Power Authority (GPA), which lists many different contact addresses,
including several specifically for T\&D. The GPA website gives access
to a diagram of the existing transmission lines, which may be able to
be imported into GIS, along with one-line diagrams showing the
electrical structure of the grid. GPA’s 2008 IRP gives data on the
relevant features of the existing generation sources (nameplate
capacity, age, heat rates, operating costs, etc.). Additionally, the
Guam Energy Office may have relevant information, as their interests
lie in promoting energy conservation, energy efficiency, and renewable
energy programs.

NREL provides access to two stations of TMY3 data that can be used for
solar and wind resource mapping, although extrapolating radiation and
wind speed data from just two points may be a challenge. NREL also
previously published a technical assessment report evaluating energy
efficiency and renewables in Guam with the Department of
Interior. This report focuses on a wide range of strategies, but does
conclude on the high potential of wind on the Guam grid, suggesting a
potential collaborative effort in sharing data related to grid
optimization.

A large amount of preprocessing of the data was required to
incorporate it into the model. The 2011-2013 load data from GPA was
averaged in order to create a model baseline year without the
fluctuations of each individual year. This was then extrapolated from
2015-2035 based on the load growth prediction from Guam's 2013 IRP
\cite{gpa13a}. 

Solar and wind resource data were extrapolated from the limited
available TMY3 data and the single meteorological tower from the
Navy. This involved combining diffuse and direct radiation to
determine incident radation levels, which could then be converted into
``sun-hours.'' Anenometer readings were extrapolated to hub heights
using the power law, with a Hellman exponent, $\alpha$, of 0.2:
\[v_2 = v_1\left(\frac{h_2}{h_1}\right)^\alpha\] 
These wind speeds could then be used to determine outputs of a GE
2.5MW turbine based on its power curve, approximated with a Gaussian
curve \cite{ge}. Each wind and solar site were designated to be
represented by either one of the three measurement sites or an average
of two, based on its proximity. While more site-specific data would
obviously be needed before construction at these sites, this serves as
a rough estimate of the available resource, given data constraints.

All load and resource data was then binned into appropriately sized
timesteps. The final model involved timesteps of 168 hours (1
week). In extrapolating resource data from 2015-2035, a uniform
degree of randomness (5\%) was included to provide variation between
different sites and years. 

\section{Results}

\section{Senstivity/Uncertainty}

\section{Discussion}

\begin{thebibliography}{99}

\bibitem{gpa14a} 
  Guam Power Authority.
  \emph{Fact Sheet}.
  Retrieved March 13, 2014 from: 
  \url{http://guampowerauthority.com/gpa_authority/about/gpa_fact_sheet.php}.

\bibitem{misty}
  Baring-Gould, I. et al. (April 2011).
  \emph{Guam Initial Technical Assessment Report}.
  National Renewable Energy Laboratory (NREL).
  Retrieved from \url{http://www.nrel.gov/docs/fy11osti/50580.pdf}

\bibitem{dsire}
  DSIRE (2012).
  \emph{Guam Renewable Energy Portfolio Goal}.
  Database of State Incentives for Renewable Energy.
  
\bibitem{gpa_trans}
  Guam Power Authority.
  \emph{GPA Transmission Islandwide Power System Single-Line Diagram}.
  Retrieved from 
  \url{http://guampowerauthority.com/gpa_authority/engineering/documents/IslandwidePlantsandTransmission.pdf}

\bibitem{gpa08}
  Guam Power Authority (2008).
  \emph{Generation Resource Handbook, FY 2008}.
  Retrieved from
  \url{http://guampowerauthority.com/gpa_authority/strategicplanning/documents/FY2008GenerationResourceHandbook.pdf}

\bibitem{eia12}
  U.S. Energy Information Administration (2012).
  \emph{Guam Territory Profile and Energy Analysis}
  Retrieved January 24, 2014 from
  \url{http://www.eia.gov/countries/country-data.cfm?fips=gq}
  
\bibitem{gpa13a}
  Guam Power Authority (Feb 2013).
  \emph{Integrated Resource Plan, FY 2013}.

\bibitem{ge}
  General Electric (GE).
  \emph{2.5MW Wind Turbine Series}.
  GE Power \& Water, Renewable Energy.
  Retrieved March 10, 2014 from
  \url{http://www.ge-energy.com/content/multimedia/_files/downloads/GEA17007A-Wind25Brochure.pdf}

\end{thebibliography}

\end{document}